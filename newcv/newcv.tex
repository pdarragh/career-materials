\RequirePackage{fix-cm}
\documentclass{article}

\usepackage[margin=0.75in]{geometry}

\usepackage[fontsize=11pt]{fontsize}
\usepackage{microtype}
\usepackage[T1]{fontenc}
\usepackage{lmodern} % To switch to Latin Modern
\rmfamily % To load Latin Modern Roman and enable the following NFSS
% declarations.
%
% Declare that Latin Modern Roman (lmr) should take its bold (b) and bold
% extended (bx) weight, and small capital (sc) shape, from the corresponding
% Computer Modern Roman (cmr) font, for the T1 font encoding.
\DeclareFontShape{T1}{lmr}{b}{sc}{<->ssub*cmr/bx/sc}{}
\DeclareFontShape{T1}{lmr}{bx}{sc}{<->ssub*cmr/bx/sc}{}

\usepackage{ragged2e}
\usepackage{titlesec}

\usepackage{array}
\usepackage{longtable}
\usepackage[longtable]{multirow}

\newlength{\cvcolumngap}
\newlength{\cvleftcolumnwidth}
\newlength{\cvmiddlecolumnwidth}
\newlength{\cvrightcolumnwidth}
\newlength{\cvmaincolumnwidth}
\setlength{\cvcolumngap}{0.25in}
\setlength{\cvleftcolumnwidth}{1.5in}
\setlength{\cvmiddlecolumnwidth}{2.5in}
\setlength{\cvrightcolumnwidth}{2.5in}
\setlength{\cvmaincolumnwidth}{\dimexpr\cvmiddlecolumnwidth +
  \dimexpr\cvcolumngap + \dimexpr\cvrightcolumnwidth}

\setlength{\parindent}{0pt}
\setlength{\LTleft}{\parindent}
\setlength{\LTright}{\fill}
\setlength{\tabcolsep}{0pt}

\usepackage[svgnames]{xcolor}
\usepackage{luacolor}

% Enable URLs. Remove the default ugly boxes and color links in black.
\usepackage[colorlinks=true]{hyperref}
\colorlet{cvurlcolor}{DarkBlue}
% Provide a custom underline that skips descenders.
% This code was copy/pasted from:
% https://alexwlchan.net/2017/10/latex-underlines/
\usepackage{contour}
\usepackage[normalem]{ulem}
\renewcommand{\ULdepth}{1.8pt}
\contourlength{0.8pt}
\renewcommand{\underline}[1]{%
  \uline{\phantom{#1}}%
  \llap{\contour{white}{#1}}%
}
% Provide a custom URL display.
\newcommand{\cvurl}[2][]{%
  \def\cvurlText{#1}%
  \ifx\cvurlText\empty
    \def\cvurlText{#2}%
  \else
    % Do nothing.
  \fi
  \href{#1}{\color{cvurlcolor}\underline{\cvurlText}}%
}

% NOTE: Use these for checking layout.
% \usepackage[unit=in]{fgruler}
% \usepackage[pattern=majmin,patternsize=0.25in,fullpage]{gridpapers}


%%%%%%%%%%%%%%%%%%%%%%%%%%%%%%%%%%%%%%%%%%%%%%%%%%%%%%%%%%%%%%%%%%%%%%%%%%%%%%%%
%%
%% CUSTOM COMMANDS
%%
%%%%%%%%%%%%%%%%%%%%%%%%%%%%%%%%%%%%%%%%%%%%%%%%%%%%%%%%%%%%%%%%%%%%%%%%%%%%%%%%

\renewcommand{\bullet}{\hspace{0.4em}\raisebox{0.4ex}{\scriptsize\textbullet}}
\newcommand{\textplus}{\texttt{+}}
\newcommand{\givenname}[1]{\fontsize{34pt}{37.4pt}\selectfont\textsc{\textls*[-10]{#1}}}
\newcommand{\familyname}[1]{\givenname{\textbf{#1}}}

\newenvironment{cvtable}
{\begin{longtable}{%
      @{}
      >{\RaggedLeft}
      p{\cvleftcolumnwidth}
      @{\extracolsep{\cvcolumngap}}
      p{\cvmiddlecolumnwidth}
      @{\extracolsep{\cvcolumngap}}
      >{\RaggedLeft}
      p{\cvrightcolumnwidth}
      @{}}}
  {\end{longtable}}
\newcommand{\cventrynormal}[2][]{#1 &
  \multicolumn{2}{l}{\parbox{\cvmaincolumnwidth}{#2}} \\ }
\newcommand{\cventrystar}[2][]{#1 &
  \multicolumn{2}{l}{\parbox{\cvmaincolumnwidth}{#2}} \\* }
\makeatletter
\newcommand{\cvindent}{\hspace{0.25in}}
\newcommand{\cvskip}{\@ifstar{ \\* }{ \\ }}
\newcommand{\cventry}{\@ifstar{\cventrystar}{\cventrynormal}}
\newcommand{\cvdetail}{\@ifstar{\cvdetailstar}{\cvdetailnormal}}
\makeatother
\newcommand{\cvitem}[2][]{\cventry[#1]{\;\bullet~#2}}
\newcommand{\cvdetailnormal}[3]{#1 & #2 & #3 \\ }
\newcommand{\cvdetailstar}[3]{#1 & #2 & #3 \\* }
% \newcommand{\cventry}[2][]{#1 &
%   \multicolumn{2}{l}{\parbox{\cvmaincolumnwidth}{#2}} \\ }
\newcommand{\cvright}[2][]{\cvdetail{#1}{}{#2}}
\newcommand{\cvfullwidth}[1]{\multicolumn{3}{l}{\parbox{\textwidth}{#1}} \\ }
% \newcommand{\cvsubheading}[2][]{\cvskip\cvskip\cventry*[#1]{\LARGE #2}\cvskip* }
\newcommand{\cvsubheading}[1]{\cvskip
  \multicolumn{3}{l}{\parbox{\textwidth}{\LARGE #1}} \\* \cvskip* }
% \newcommand{\cvsubsubheading}[2][]{\cventry*[#1]{\textsc{#2}}\cvskip* }
\newcommand{\cvsubsubheading}[1]{\cvfullwidth{\textsc{#1}}\cvskip* }

\usepackage{booktabs}

\usepackage{makecell}
\newcommand{\makecellr}[1]{\renewcommand\cellalign{rt}\makecell{#1}}
\newcommand{\makecelll}[1]{\renewcommand\cellalign{lt}\makecell{#1}}


%%%%%%%%%%%%%%%%%%%%%%%%%%%%%%%%%%%%%%%%%%%%%%%%%%%%%%%%%%%%%%%%%%%%%%%%%%%%%%%%
%%
%% DOCUMENT
%%
%%%%%%%%%%%%%%%%%%%%%%%%%%%%%%%%%%%%%%%%%%%%%%%%%%%%%%%%%%%%%%%%%%%%%%%%%%%%%%%%

\begin{document}

\begin{cvtable}

  % \cline{1-3}
  \cvdetail{\multirow{2}{=}{\RaggedLeft\givenname{Pierce}}}
  {\multirow{2}{=}{\familyname{Darragh}}}
  {\raisebox{-3pt}{\cvurl[pierce.darragh@gmail.com]{mailto:pierce.darragh@gmail.com}}}
  \cvright{\cvurl{https://pdarragh.github.io}}
  % \cline{1-3}

  % \cvskip\cvskip

  % \cvfullwidth{My research focuses on the design of programming language
  %   features that improve the ability of developers to translate their thoughts
  %   into code. I am particularly interested in advanced ergonomic static
  %   analyses.}
  \cvskip

  \cvsubheading{Education}

  \cventry*[In-Progress]{\textbf{University of Maryland}}
  \cventry*{\cvindent{}PhD in Computer Science (Programming Languages).}
  \cventry{\cvindent{}Advised by David Van Horn.}

  \cvskip

  \cventry*[2018]{\textbf{University of Utah}}
  \cventry*{\cvindent{}MS in Computer Science.}
  \cventry{\cvindent{}BS in Computer Science, Minor in Linguistics.}

  \cvsubheading{Teaching}

  \cvsubsubheading{As Instructor}

  \cventry*[Spring 2022]{\textbf{CMSC 388X: Introduction to Programming Language
      Theory}}
  \cventry{\cvurl{https://www.cs.umd.edu/class/spring2022/cmsc388X/}}
  \cventry{I developed a new undergraduate course to teach undergraduate
    students basic concepts in programming language theory. On one day of the
    week we covered content roughly lifted from \textit{Types and Programming
      Languages}, and on the other day of the week we had discussions on
    assigned papers. Covered topics included:}
  \cvitem{Syntactic theory (e.g., BNF grammars, metafunctions).}
  \cvitem{Structural induction over syntax for constructing proofs.}
  \cvitem{Reduction and typing relations via small-step operational semantics.}
  \cvitem{The lambda calculus.}
  \cvitem{Extending the lambda calculus with types and recursion.}
  \cventry{To select the papers we read, students formed small groups and each
    group had to choose a paper from a \cvurl[pre-approved
    list]{https://www.cs.umd.edu/class/spring2022/cmsc388X/Papers.html} to read
    and present for discussion with the class. Student feedback for the class
    was overwhelmingly positive.}

  \cvskip

  \cvsubsubheading{As Graduate Teaching Assistant}

  \cventry*[Fall 2022--Present]{\textbf{CMSC 430: Compilers}}
  \cventry{\cvurl{https://www.cs.umd.edu/class/spring2025/cmsc430/}}
  \cventry{This course teaches students how to implement compilers in Racket for
    languages of increasing complexity, targeting the x86 assembly language.}
  \cvskip
  \cventry{Over the past eight semesters, I have worked with various course
    instructors (Professors David Van Horn, José Manuel Calderón Trilla,
    Milijana Surbatovich, and Anwar Mamat), to improve the course. My work has
    included:}
  \cvitem{Creating new assignments and modifying existing ones.}
  \cvitem{Writing and grading midterm exams.}
  \cvitem{Developing new instructional material (lecture notes, quizzes, etc.).}
  \cvitem{Implementing new automated grading infrastructure.}
  \cvitem{Analyzing course data to inform subsequent decisions and discussions.}
  \cventry{I have also been working on the \underline{a86 Assembly Interpreter}
    (see Selected Projects), which I intend to use for this class to guide
    students' debugging efforts in a more systematic and course-specific
    manner.}

  \cvskip

  \cventry*[2023--2024]{\textbf{Excellence in Teaching Award}}
  \cventry{I was anonymously nominated for --- and subsequently selected to win
    --- a departmental Excellence in Teaching Award for the 2023--2024 academic
    year due to my efforts in TAing CMSC 430: Compilers (above). The department
    chooses \sim 5 recipients for these awards each year, selected from among
    all staff and faculty.}

  \cvskip

  \cventry*[Spring 2023]{\textbf{CMSC 433: Programming Paradigms}}
  \cventry{This course was loosely based on the previous semester's CMSC 488B
    (below), which taught students how to use Haskell by thinking lazily and
    functionally.}

  \cvskip

  \cventry*[Spring 2022]{\textbf{CMSC 488B: Advanced Functional Programming}}
  \cventry{This course taught students how to program in Haskell, including
    discussions of basic category theory and the use of QuickCheck.}

  \cvskip

  \cventry*[Fall 2021]{\textbf{CMSC 330: Programming Languages}}
  \cventry{This required undergraduate course taught students about programming
    in Ruby, OCaml, and Rust. It also introduced students to concepts in basic
    programming language theory, including the lambda calculus, operational
    semantics, type-checking, parsing, and so on.}

  \cvsubheading{Research}

  \cvsubsubheading{Publications}

  \cventry*[GPCE 2023]{\textbf{Generating Conforming Programs With Xsmith.}}
  \cventry{
    \begin{tabular}{>{\RaggedLeft}p{0.75in}@{\extracolsep{0.125in}}p{4.37in}}
      Authors:  & William Gallard Hatch, \underline{Pierce Darragh}, Sorawee
                  Porncharoenwase, Guy Watson, and Eric Eide. \\
      Date:     & October 2023. \\
      Venue:    & International Conference on Generative Programming: Concepts \&
                  Experiences 2023. \\
      URL:      & \cvurl{pdarragh.github.io/p/gpce23} \\
      Synopsis: & Xsmith is a domain-specific language for implementing fuzzers
                  that operate in the style of Csmith, implemented in Racket. We
                  provide implementations for a handful of languages and report
                  on bugs identified in some of their implementations. \\
    \end{tabular}}

  \cvskip

  \cventry*[BRM 2021]{\textbf{SweetPea: A standard language for factorial
      experimental design.}}
  \cventry{
    \begin{tabular}{>{\RaggedLeft}p{0.75in}@{\extracolsep{0.125in}}p{4.37in}}
      Authors:  & Sebastian Musslick, Anastasia Cherkaev, Ben Draut, Ahsan
                  Sajjad Butt, \underline{Pierce Darragh}, Vivek Srikumar,
                  Matthew Flatt, and Jonathan D Cohen. \\
      Date:     & April 2021. \\
      Venue:    & Behavior Research Methods, volume 54, issue 2. \\
      URL:      & \cvurl{pdarragh.github.io/p/sweetpea} \\
      Synopsis: & We introduce SweetPea, a domain-specific language for
                  specifying factorial experimental designs, implemented in
                  Python. Although built with the field of psychology in mind,
                  SweetPea can be used for most factorial experiments. \\
    \end{tabular}}

  \cvskip

  \cventry*[Scheme 2020]{\textbf{Clotho: A Racket Library for Parametric
      Randomness.}}
  \cventry{
    \begin{tabular}{>{\RaggedLeft}p{0.75in}@{\extracolsep{0.125in}}p{4.37in}}
      Authors:  & \underline{Pierce Darragh}, William Gallard Hatch, and Eric
                  Eide. \\
      Date:     & August 2020. \\
      Venue:    & Scheme and Functional Programming Workshop 2020. \\
      URL:      & \cvurl{pdarragh.github.io/p/scheme20} \\
      Synopsis: & Clotho is a Racket library that implements \textit{parametric
                  randomness}, a style of (pseudo)random generation where
                  external manipulations of recorded sampling events correspond
                  to discrete changes in the structure of the output. It was
                  built as part of the implementation of Xsmith. \\
    \end{tabular}}

  \cvskip

  \cventry*[ICFP 2020]{\textbf{Parsing with Zippers (Functional Pearl).}}
  \cventry{
    \begin{tabular}{>{\RaggedLeft}p{0.75in}@{\extracolsep{0.125in}}p{4.37in}}
      Authors:  & \underline{Pierce Darragh} and Michael D. Adams. \\
      Date:     & August 2020. \\
      Venue:    & PACMPL, volume 4, issue ICFP. \\
      URL:      & \cvurl{pdarragh.github.io/p/icfp20} \\
      Synopsis: & Parsing with Derivatives is a known technique for implementing
                  a parser with an elegant theory, but which suffers from poor
                  performance. Parsing with Zippers is built upon the same
                  theory of parsing, but featuring a deviation in the mode of
                  traversal of the input that produces a significant speedup. \\
    \end{tabular}}

  \cvskip

  \cvsubsubheading{Presentations}

  \cventry[RacketCon 2020]{\textbf{Clotho: A Racket Library for Parametric
      Randomness.}}
  \cventry{I was invited to give this talk again after presenting at the Scheme
    Workshop.}
  \cventry[Scheme 2020]{\textbf{Clotho: A Racket Library for Parametric
      Randomness.}}
  \cventry[ICFP 2020]{\textbf{Parsing with Zippers (Functional Pearl).}}

  \cvskip

  \cvsubsubheading{Selected Projects}

  \cvdetail*{In-Progress}{\textbf{a86 Assembly
      Interpreter}}{\cvurl{github.com/cmsc430/a86-interpreter}}
  \cventry{CMSC 430 has students implement compilers in Racket targeting a
    restricted subset of the x86-64 assembly language, which we call ``a86.'' I
    am implementing a step-able, time-traveling a86 interpreter with helpful,
    course-tailored implementation details to improve the student debugging
    experience.}
  \cvskip
  \cventry{Similar to efforts like Python Tutor and Learn-OCaml, I intend to
    extend the capabilities of this interpreter to provide automated
    course-specific feedback for students, such as hints for specific
    assignments based on heuristics we use effectively as instructors already.
    Additionally, I am developing a mechanism to synthesize information about
    student submissions to our automated grading platform to support instructor
    feedback in a setting with hundreds of students and too-few instructors.}

  \cvskip

  \cvdetail*{2020--2021}{\textbf{SweetPea}}{\cvurl{sweetpea-org.github.io}}
  \cventry{A domain-specific language built for the declarative specification of
    randomized experimental designs. I rewrote the back-end processing system
    and revised the front-end API.}

  \cvskip

  \cvdetail*{2019--2020}{\textbf{Xsmith}}{\cvurl{www.flux.utah.edu/project/xsmith}}
  \cventry{A generic fuzzer generator, built in the spirit of Csmith but
    implemented as a domain-specific language in Racket. I implemented the
    Python fuzzer specification and its necessary internal components, and also
    developed a new Racket library (named Clotho) to improve Xsmith's
    capabilities for exploring state spaces.}

  \cvsubheading{Professional Experience}

  \cventry*[2021--Present]{\textbf{University of Maryland}, Graduate Teaching
    Assistant}
  \cventry{Assisted in the instruction and execution of various courses during
    my PhD.}

  \cvskip

  \cventry*[2020--2021]{\textbf{SweetPea Research Group, University of Utah},
    Research Associate}
  \cventry{Rewrote implementation and expended functionality of
    \underline{SweetPea} under the direction of Matthew Flatt and in
    coordination with a team at Princeton University. This project culminated in
    a publication.}

  \cvskip

  \cventry*[2019--2020]{\textbf{Flux Research Group, University of Utah},
    Research Associate}
  \cventry{Developed new features for \underline{Xsmith} under the direction of
    Eric Eide, including support for alternate type systems and a random program
    generator for Python. Also developed a new library, \underline{Clotho}, to
    enable repeatable complex random generation simulation. This work resulted
    in two publications, one each for Xsmith and Clotho.}

  \cvskip

  \cventry*[2018--2019]{\textbf{U-Combinator Research Group, University of
      Utah}, Research Associate}
  \cventry{Worked with Michael Adams on various projects as an extension of
    research that had been started as an undergraduate. This work resulted in
    the publication of \underline{Parsing with Zippers}.}

  \cvskip

  \cventry*[Summer 2017]{\textbf{Apple, Inc.}, Software Engineer Intern}
  \cventry{Designed, built, and presented a secure framework for automatically
    creating proxy servers intended for use in internal penetration testing.}

  \cvsubheading{Academic Awards}

  \cventry[2023--2024]{Excellence in Teaching, University of Maryland Department
    of Computer Science.}
  \cventry[2021--Present]{Dean's Fellowship, University of Maryland.}
  \cventry[2012--2016]{National Merit Scholarship, sponsored by
    E{\textasteriskcentered}TRADE.}
  \cventry[2012]{Merit Scholarship with Presidential Honors, University of
    Utah.}

  \cvsubheading{Non-Academic Service and Leadership}

  \cventry[2020--2022]{Moderator, /r/ProgrammingLanguages Discord server.}
  \cventry[2020--2021]{Community manager, Jean Yang's \texttt{\#PLTalk} Twitch
    stream and Discord server.}
  \cventry[2014--2017]{Web administrator, University of Utah Club Swim Team.}
  \cventry[2014--2015]{Men's team captain, University of Utah Club Swim Team.}
\end{cvtable}

\end{document}
