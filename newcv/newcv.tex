\RequirePackage{fix-cm}
\documentclass{article}

\usepackage[margin=0.75in]{geometry}

\usepackage[fontsize=11pt]{fontsize}
\usepackage{microtype}
\usepackage[T1]{fontenc}
\usepackage{lmodern} % To switch to Latin Modern
\rmfamily % To load Latin Modern Roman and enable the following NFSS declarations.
% Declare that Latin Modern Roman (lmr) should take
% its bold (b) and bold extended (bx) weight, and small capital (sc) shape,
% from the corresponding Computer Modern Roman (cmr) font, for the T1 font encoding.
\DeclareFontShape{T1}{lmr}{b}{sc}{<->ssub*cmr/bx/sc}{}
\DeclareFontShape{T1}{lmr}{bx}{sc}{<->ssub*cmr/bx/sc}{}

\usepackage{ragged2e}
\usepackage{titlesec}

\usepackage{array}
\usepackage{longtable}
\usepackage[longtable]{multirow}

\newlength{\cvcolumngap}
\newlength{\cvleftcolumnwidth}
\newlength{\cvmiddlecolumnwidth}
\newlength{\cvrightcolumnwidth}
\newlength{\cvmaincolumnwidth}
\setlength{\cvcolumngap}{0.25in}
\setlength{\cvleftcolumnwidth}{1.5in}
\setlength{\cvmiddlecolumnwidth}{2.5in}
\setlength{\cvrightcolumnwidth}{2.5in}
\setlength{\cvmaincolumnwidth}{\dimexpr\cvmiddlecolumnwidth +
  \dimexpr\cvcolumngap + \dimexpr\cvrightcolumnwidth}

\setlength{\parindent}{0pt}
\setlength{\LTleft}{\parindent}
\setlength{\LTright}{\fill}
\setlength{\tabcolsep}{0pt}

\usepackage[svgnames]{xcolor}

% Enable URLs. Remove the default ugly boxes and color links in black.
\usepackage[colorlinks=true]{hyperref}
\colorlet{cvurlcolor}{DarkBlue}
% Provide a custom underline that skips descenders.
% This code was copy/pasted from:
% https://alexwlchan.net/2017/10/latex-underlines/
\usepackage{contour}
\usepackage[normalem]{ulem}
\renewcommand{\ULdepth}{1.8pt}
\contourlength{0.8pt}
\renewcommand{\underline}[1]{%
  \uline{\phantom{#1}}%
  \llap{\contour{white}{#1}}%
}
% Provide a custom URL display.
\newcommand{\cvurl}[2][]{%
  \def\cvurlText{#1}%
  \ifx\cvurlText\empty
    \def\cvurlText{#2}%
  \else
    % Do nothing.
  \fi
  \href{#1}{\color{cvurlcolor}\underline{\cvurlText}}%
}

% NOTE: Use these for checking layout.
% \usepackage[unit=in]{fgruler}
% \usepackage[pattern=majmin,patternsize=0.25in,fullpage]{gridpapers}


%%%%%%%%%%%%%%%%%%%%%%%%%%%%%%%%%%%%%%%%%%%%%%%%%%%%%%%%%%%%%%%%%%%%%%%%%%%%%%%%
%%
%% CUSTOM COMMANDS
%%
%%%%%%%%%%%%%%%%%%%%%%%%%%%%%%%%%%%%%%%%%%%%%%%%%%%%%%%%%%%%%%%%%%%%%%%%%%%%%%%%

\renewcommand{\bullet}{\hspace{0.4em}\raisebox{0.4ex}{\scriptsize\textbullet}}
\newcommand{\textplus}{\texttt{+}}
\newcommand{\givenname}[1]{\fontsize{34pt}{37.4pt}\selectfont\textsc{\textls*[-10]{#1}}}
\newcommand{\familyname}[1]{\givenname{\textbf{#1}}}

\newenvironment{cvtable}
{\begin{longtable}{%
      @{}
      >{\RaggedLeft}
      p{\cvleftcolumnwidth}
      @{\extracolsep{\cvcolumngap}}
      p{\cvmiddlecolumnwidth}
      @{\extracolsep{\cvcolumngap}}
      >{\RaggedLeft}
      p{\cvrightcolumnwidth}
      @{}}}
  {\end{longtable}}
\newcommand{\cventrynormal}[2][]{#1 &
  \multicolumn{2}{l}{\parbox{\cvmaincolumnwidth}{#2}} \\ }
\newcommand{\cventrystar}[2][]{#1 &
  \multicolumn{2}{l}{\parbox{\cvmaincolumnwidth}{#2}} \\* }
\makeatletter
\newcommand{\cvindent}{\hspace{0.25in}}
\newcommand{\cvskip}{\@ifstar{ \\* }{ \\ }}
\newcommand{\cventry}{\@ifstar{\cventrystar}{\cventrynormal}}
\newcommand{\cvdetail}{\@ifstar{\cvdetailstar}{\cvdetailnormal}}
\makeatother
\newcommand{\cvitem}[2][]{\cventry[#1]{\bullet~#2}}
\newcommand{\cvdetailnormal}[3]{#1 & #2 & #3 \\ }
\newcommand{\cvdetailstar}[3]{#1 & #2 & #3 \\* }
% \newcommand{\cventry}[2][]{#1 &
%   \multicolumn{2}{l}{\parbox{\cvmaincolumnwidth}{#2}} \\ }
\newcommand{\cvright}[2][]{\cvdetail{#1}{}{#2}}
\newcommand{\cvfullwidth}[1]{\multicolumn{3}{l}{\parbox{\textwidth}{#1}} \\ }
\newcommand{\cvsubheading}[2][]{\cvskip\cvskip\cventry*[#1]{\LARGE
    #2}\cvskip* }
\newcommand{\cvsubsubheading}[2][]{\cventry*[#1]{\textsc{#2}}\cvskip* }

\usepackage{booktabs}


%%%%%%%%%%%%%%%%%%%%%%%%%%%%%%%%%%%%%%%%%%%%%%%%%%%%%%%%%%%%%%%%%%%%%%%%%%%%%%%%
%%
%% DOCUMENT
%%
%%%%%%%%%%%%%%%%%%%%%%%%%%%%%%%%%%%%%%%%%%%%%%%%%%%%%%%%%%%%%%%%%%%%%%%%%%%%%%%%

\begin{document}

\begin{cvtable}

  % \cline{1-3}
  \cvdetail{\multirow{2}{=}{\RaggedLeft\givenname{Pierce}}}
  {\multirow{2}{=}{\familyname{Darragh}}}
  {\raisebox{-3pt}{\cvurl[pierce.darragh@gmail.com]{mailto:pierce.darragh@gmail.com}}}
  \cvright{\cvurl[pdarragh.github.io]{https://pdarragh.github.io}}
  % \cline{1-3}

  \cvskip\cvskip

  \cvfullwidth{My research focuses on the design of programming language
    features that improve the ability of developers to translate their thoughts
    into code. I am particularly interested in advanced ergonomic static
    analyses.}

  \cvsubheading{Education}

  \cventry*[In-Progress]{\textbf{University of Maryland}}
  \cventry*{\cvindent{}PhD in Computer Science.}
  \cventry{\cvindent{}Advised by David Van Horn.}

  \cvskip

  \cventry*[2018]{\textbf{University of Utah}}
  \cventry*{\cvindent{}MS in Computer Science.}
  \cventry{\cvindent{}BS in Computer Science, Minor in Linguistics.}

  \cvsubheading{Research}

  \cvsubsubheading{Publications}

  \cventry[BRM 2021]{\textbf{SweetPea: A standard language for factorial experimental design.}}
  \cventry{
    \begin{tabular}{>{\RaggedLeft}p{0.75in}@{\extracolsep{0.125in}}p{4.37in}}
      Authors: & Sebastian Musslick, Anastasia Cherkaev, Ben Draut, Ahsan
                 Sajjad Butt, \underline{Pierce Darragh}, Vivek Srikumar,
                 Matthew Flatt, Jonathan D Cohen. \\
      Date:    & April 2021. \\
      Venue:   & Behavior Research Methods, volume 54, issue 2. \\
      URL:     & \cvurl{pdarragh.github.io/p/sweetpea} \\
    \end{tabular}}

  \cvskip

  \cventry[Scheme 2020]{\textbf{Clotho: A Racket Library for Parametric
      Randomness.}}
  \cventry{
    \begin{tabular}{>{\RaggedLeft}p{0.75in}@{\extracolsep{0.125in}}p{4.37in}}
      Authors: & \underline{Pierce Darragh}, William Hatch, and Eric Eide. \\
      Date:    & August 2020. \\
      Venue:   & Scheme and Functional Programming Workshop 2020. \\
      URL:     & \cvurl{pdarragh.github.io/p/scheme20} \\
    \end{tabular}}

  \cvskip

  \cventry[ICFP 2020]{\textbf{Parsing with Zippers (Functional Pearl).}}
  \cventry{
    \begin{tabular}{>{\RaggedLeft}p{0.75in}@{\extracolsep{0.125in}}p{4.37in}}
      Authors: & \underline{Pierce Darragh} and Michael D. Adams. \\
      Date:    & August 2020. \\
      Venue:   & PACMPL, volume 4, issue ICFP. \\
      URL:     & \cvurl{pdarragh.github.io/p/icfp20} \\
    \end{tabular}}

  \cvskip

  \cvsubsubheading{Presentations}

  \cventry[RacketCon 2020]{\textbf{Clotho: A Racket Library for Parametric
      Randomness.}}
  \cventry[Scheme 2020]{\textbf{Clotho: A Racket Library for Parametric
      Randomness.}}
  \cventry[ICFP 2020]{\textbf{Parsing with Zippers (Functional Pearl).}}

  \cvskip

  \cvsubsubheading{Selected Projects}

  \cvdetail*{In-Progress}{\textbf{a86 Assembly
      Interpreter}}{\cvurl{github.com/cmsc430/a86-interpreter}}
  \cventry{UMD uses a restricted subset of the x86-64 assembly language (called
    a86) for their undergraduate compilers course, which is implemented in
    Racket. I am implementing a step-able interpreter with helpful feedback to
    improve the student debugging experience.}

  \cvskip

  \cvdetail*{2020--2021}{\textbf{SweetPea}}{\cvurl{sweetpea-org.github.io}}
  \cventry{A domain-specific language built for the declarative specification of
    randomized experimental designs. I rewrote the back-end processing system
    and revised the front-end API.}

  \cvskip

  \cvdetail*{2019--2020}{\textbf{Xsmith}}{\cvurl{www.flux.utah.edu/project/xsmith}}
  \cventry{A generic fuzzer generator, built in the spirit of Csmith but
    implemented as a domain-specific language in Racket. I implemented the
    Python fuzzer specification and its necessary internal components, and also
    developed a new Racket library (named Clotho) to improve Xsmith's
    capabilities for exploring state spaces.}

  \cvsubheading{Teaching}

  \cvsubsubheading{Instructor}

  \cventry*[Spring 2022]{\textbf{CMSC 388X: Introduction to Programming Language
      Theory}}

  \cventry{I developed a new undergraduate course with a labmate. Select topics
    included:}
  \cvitem{Syntactic theory (e.g., BNF grammars, metafunctions).}
  \cvitem{Structural induction over syntax for proofs.}
  \cvitem{Reduction and typing relations via small-step operational semantics.}
  \cvitem{The lambda calculus.}
  \cvitem{Extending the lambda calculus with types and recursion.}
  \cventry{Students also formed small groups, and each group selected one paper
    from a pre-approved list to read and present for discussion with the class.}

  \cvskip

  \cvsubsubheading{Graduate Teaching Assistant}

  \cventry[Spring 2023]{\textbf{Compilers} --- CMSC 430}

  \cventry[Fall 2022]{\textbf{Compilers} --- CMSC 430}

  \cventry[Spring 2022]{\textbf{Advanced Functional Programming} --- CMSC 488B}

  \cventry[Fall 2021]{\textbf{Programming Languages} --- CMSC 330}

  \cvsubheading{Academic Service}

  \cventry[2019--2021]{Organizer, programming languages reading group,
    University of Utah.}
  \cventry[2013 \& 2014]{Volunteer judge, Salt Lake Valley Science and
    Engineering Fair.}

  \cvsubheading{Awards}

  \cventry[2021--Present]{Dean's Fellowship, University of Maryland.}
  \cventry[2012--2016]{National Merit Scholarship, sponsored by
    E{\textasteriskcentered}TRADE.}
  \cventry[2012]{Merit Scholarship with Presidential Honors, University of
    Utah.}

  \cvsubheading{Industry Experience}

  \cventry[Summer 2017]{\textbf{Apple, Inc.}, Software Engineer Intern}
  \cventry{Designed, built, and presented a secure framework for automatically
    creating proxy servers intended for use in internal penetration testing.}

  \cvsubheading{Non-Academic Service and Leadership}

  \cventry[2020--Present]{Moderator, /r/ProgrammingLanguages Discord server.}
  \cventry[2020--2021]{Community manager, Jean Yang's \texttt{\#PLTalk} Twitch
    stream and Discord server.}
  \cventry[2014--2017]{Web administrator, University of Utah Club Swim Team.}
  \cventry[2014--2015]{Men's team captain, University of Utah Club Swim Team.}
\end{cvtable}

\end{document}